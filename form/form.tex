\documentclass{form}
\usepackage{hyperref}
\author{Diogo Rodrigues (\texttt{\href{mailto:up201806429@fe.up.pt}{up201806429@fe.up.pt}})}
\title{RCOM@FEUP 2020/21 -- Form}
\renewcommand\rightmark{
    Licensed under \href{https://creativecommons.org/licenses/by-nc-nd/4.0/}{CC BY-NC-ND 4.0 \vcenteredinclude{by-nc-nd-eu}}
}

\setlength{\columnsep}{0.5em}

% Document
\begin{document}
\sisetup{range-phrase=-}
\sisetup{range-units=single}

\maketitle

\begin{minipage}{0.425\textwidth}
    \section*{Introduction}
    \paragraph{Circuit-switched} Require dedicated end-to-end connections with dedicated channel at all times.
    \paragraph{Packet-switched} Data is split into packets, which are exchanged link-by-link and assembled in endpoints; connectionless.
\end{minipage}
\begin{minipage}{38em}
    % \section*{Introduction to networks}
    \begin{tabular}{@{}l | l | l | p{5.4em}@{}}
        \textbf{Inter-CPU dist.} & \textbf{CPU in same} & \textbf{Type of network} & \textbf{Examples} \\ \hline
        $\SI{    1}{     \meter}$ & Sqr. meter   & Personal Area Network (PAN)               & Bluetooth \\ \hline
        $\SI{   10}{     \meter}$ & Room         & \multirow{3}{*}{Local Area Network (LAN)} & \multirow{3}{5.4em}{Over switch network} \\ \cline{1-2}
        $\SI{  100}{     \meter}$ & Building     &                                           & \\ \cline{1-2}
        $\SI{    1}{\kilo\meter}$ & Campus       &                                           & \\ \hline
        $\SI{   10}{\kilo\meter}$ & City         & Metrop. Area Network (PAN)                & Cable TV \\ \hline
        $\SI{  100}{\kilo\meter}$ & Country      & \multirow{2}{*}{Wide Area Network (PAN)}  & \multirow{2}{5.4em}{ISP network} \\ \cline{1-2}
        $\SI{ 1000}{\kilo\meter}$ & Continent    &                                           & \\ \hline
        $\SI{10000}{\kilo\meter}$ & Planet       & Internet                                  & \\
    \end{tabular}
\end{minipage}

\begin{minipage}{0.06\textwidth}
    \subsection*{OSI model}
\end{minipage}%
\begin{minipage}{0.93\textwidth}
    \begin{tabular}{@{}p{9mm} | l | l | l | l | l@{}}
        \multicolumn{3}{c|}{\textbf{Layer}}                    & Data unit                & Function                                                        & Protocols             \\ \hline
        \multirow{4}{9mm}{Host layers}      & 7 & Application  & \multirow{3}{*}{Data}    & High-level APIs (resource sharing, remote file access)          & DHCP, DNS, FTP, HTTP, RIP                      \\ \cline{2-3} \cline{5-6}
                                            & 6 & Presentation &                          & Data translation (encoding, compression, encrypt/decrypt)       &                       \\ \cline{2-3} \cline{5-6}
                                            & 5 & Session      &                          & Managing communication sessions                                 &                       \\ \cline{2-6}
                                            & 4 & Transport    & Segment                  & Reliable segment transmission (segmentation, ACK)               & TCP,UDP               \\ \hline
        \multirow{3}{9mm}{Media layers}     & 3 & Network      & Packet                   & Multi-node network (addressing, routing, traffic control)       & IP, ICMP              \\ \cline{2-6}
                                            & 2 & Data link    & Frame                    & Reliable frame transmission between two nodes                   & ARP,MAC,Ethernet, PPP \\ \cline{2-6}
                                            & 1 & Physical     & Bit, symb.               & Communication of raw bits over physical medium                  &                       \\
    \end{tabular}
\end{minipage}%

\vspace{0em}\rule{\textwidth}{1.0pt}\vspace{-0em}

\begin{minipage}[c]{0.46\textwidth}
    \section*{Physical layer}

    Each communication media has its own transference function $H(f)$ which impacts the original signal $S(f)$ according to $R(f)=H(f) S(f)$.

    \paragraph{Nyquist theorem}
    A sampler operating at frequency $f_s$ can completely reconstruct a signal with bandwidth $B$ when
    $f_s > 2B$

    \paragraph{Nyquist bitrate}
    The theoretical maximum capacity of a noiseless channel with $M$ signal levels is $C = 2 B \log_2{M}$

    \begin{tabular}{@{}p{4.1em} | p{23.6em}@{}}
        \textbf{Transm.} & \textbf{Description} \\ \hline
        Baseband & Signal has frequencies from $\SI{0}{\hertz}$ to $B$; wires \\ \hline
        Passband & Signal uses a (usually small) band of frequencies $[f_1,f_2]$ around the carrier wave frequency $f_c$; wireless/optical
    \end{tabular}

    \subsection*{Baseband transmission codes}
    \begin{tabular}{@{}p{3.6em} | p{19.4em}@{}}
        \textbf{NRZ-L} & Non-return-to-Zero level: two levels for 0/1 \\ \hline
        \textbf{NRZ-I} & Non-return-to-Zero inverted: lvl change = 1 \\ \hline
        \textbf{Manch-ester} & Used in Ethernet; $+\rightarrow-$ is 1, $-\rightarrow+$ is 0
    \end{tabular}

    \paragraph{Clock recovery}
    Bursts of same symbol may confuse the receiver. Manchester naturally solves it; other codings can use 4B/5B maps (4 bits are coded into 5 bits for transmission).

\end{minipage}
\begin{minipage}[c]{0.33\textwidth}
    \subsection*{Modulations}
    $s(t)$ -- Signal function to be transmitted

    \begin{tabular}{@{}l | p{13.0em}@{}}
        \textbf{Amplitude}  & $s(t) = A_i \cos{(2\pi f_c t)}$ \\ \hline
        \textbf{Phase}      & $s(t) = A \cos{(\theta_i + 2\pi f_c t)}$ \\ \hline
        \textbf{Quadrature} & $s(t) = A_i \cos{(\theta_i + 2\pi f_c t)}$; M-QAM (quadrature amplitude modulation), uses $M$ symbols
    \end{tabular}

    \paragraph{Shannon's Law}
    The max data transmission rate $C$ over a channel in the presence of noise is
    \begin{equation*}
        C = B \log_2{\left(1+\frac{P_r}{N_0 B}\right)}
    \end{equation*}

    \subsection*{Guided transmission}
    \begin{tabular}{@{}l | p{3.8em} | p{3.5em} | p{3.7em}@{}}
        \textbf{Cable} & \textbf{B ($\SI{}{\giga\hertz}$)} & \textbf{Atten. ($\SI{}{\deci B / \kilo\meter}$)} & \textbf{Delay ($\SI{}{\micro\second / \kilo\meter}$)} \\ \hline
        UTP            & $0.1-0.6        $ & $20-250$       & 5 \\
        Coaxial        & $\approx 1      $ & $\approx 150$  & 4 \\
        Fiber optics   & $\approx 30\,000$ & $< 1$          & 5 \\
    \end{tabular}
\end{minipage}
\begin{minipage}[c]{0.20\textwidth}

    \subsection*{Gain and attenuation}

    \begin{itemize}
        \setlength\itemsep{-0.2em}
        \item $P/\SI{}{\deci B} = P/\SI{}{\deci B \watt} = 10 \log_{10}{(P/\SI{}{\watt})}$
        \item $P/\SI{}{\deci B \milli} = 10 \log_{10}{(P/\SI{}{\milli\watt})}$
        \item $Gain/\SI{}{\deci B} = -Attenuation/\SI{}{\deci B}$
    \end{itemize}

    \subsection*{Wireless transmission}
    VLF (very low frequency), LF, MF follow Earth's curvature, HF bounce off the ionosphere.

    \paragraph{Free space loss}
    Two ideal isotropic antennas distanced $d$,
    \begin{equation*}
        \frac{P_t}{P_r} = \frac{(4\pi d)^2}{\lambda ^2} = \frac{(4\pi f d)^2}{c^2}
    \end{equation*}
\end{minipage}

\vspace{-0em}\rule{\textwidth}{1.0pt}

\begin{minipage}[c]{0.41\textwidth}
    \section*{Data link layer}

    \paragraph{Byte stuffing}
    An escape character is chosen, and flags/escape chars are encoded to avoid ambiguity.

    \subsection*{Error detection}
    \paragraph{Parity check}
    One parity bit is added every $k$ bits; simple but does not detect even number of errors.

    \paragraph{Bidimensional parity}
    Any 4 errors in rectangular configuration are undetectable.

    \paragraph{Cyclic Redundancy Check (CRC)}
    given encoding parameters $r \in \mathbb{N}$ and $G$ a number with $r+1$ bits, encodes message $M$ by adding the remainer $R$  of $M \times 10^{r}$ divided by $G$ (where $M \times x^r \equiv A\times G + R \pmod{2}$ if everything is interpreted as a polynomial) to the end of $M$ ($R$ always has $r$ bits).


    \subsection*{Automatic Repeat Request (ARQ)}
    \paragraph{Stop\&Wait}
    Each received frame must be ACK, sender only sends next frame if all previous frames were acknowledged.
    \begin{center}
        \begin{tabular}{c c}
            $\displaystyle a = \frac{T_{prop}}{T_f}$ & $\displaystyle P[A=k] = p_e^{k-1} (1-p_e)$ \\
            \multicolumn{2}{c}{$\displaystyle \expected[A] = \sum_{k=1}^{\infty}{k \times P[A=k]} = \frac{1}{1-p_e}$} \\
            \multicolumn{2}{c}{$\displaystyle S            = \frac{T_f}{E[A](T_f+2T_{prop})} = \frac{1}{E[A](1+2a)} = \frac{1-p_e}{1+2a}$}
        \end{tabular}
    \end{center}
\end{minipage}
\begin{minipage}[c]{0.58\textwidth}
    \begin{minipage}[t]{0.60\linewidth}
        \paragraph{Go Back N}
        Allows transmission before previous frames were ACK. It sends ACK(NR) meaning it acknowledges all packets with $index < NR$.
        If an out-of-sequence frame is received, REJ is sent with the expected frame number on the first time; subsequent out-of-sequence frames are silently rejected.
        \vspace{-1em}
        \begin{gather*}
            W = M-1 = 2^k-1 \\
            S = \begin{dcases}
                    \frac{1-p_e}{1+2ap_e}               & : W \geq 1+2a \\
                    \frac{W(1-p_e)}{(1+2a)(1-p_e+Wp_e)} & : W < 1+2a
                \end{dcases}
        \end{gather*}
    \end{minipage}~
    \begin{minipage}[t]{0.38\linewidth}
        \paragraph{Selective Repeat}
        Similar to Go Back N, but receiver accepts out-of-sequence frames and notifies the sender to send missing frames only.
        \begin{gather*}
            W = M/2 = 2^{k-1} \\
            S = \begin{dcases}
                1-p_e                 & : W \geq 1+2a \\
                \frac{W(1-p_e)}{1+2a} & : W < 1+2a
            \end{dcases}
        \end{gather*}
    \end{minipage}

    \subsection*{Reliability in TCP/IP reference model}
    

    $PLR$ -- Packet loss ratio; $K$ -- number of links between sender and receiver; assume all links have same $C$ and $PLR$.
    \begin{center}
        \begin{tabular}{@{}p{24mm} | l@{}}
            \textbf{Strategy} & \textbf{Description} \\ \hline
            Link-by-Link & \multirow{3}{24.0em}{On error, the station closest to the sender notifies it. Repairs losses link by link. More complex, but better on very unreliable media.} \\
            $S=1-PLR$ \\
            \\
            End-to-End   & \multirow{2}{24.0em}{On error, the receiver notifies the sender. Less complex, but not acceptable on very unreliable media.} \\
            $S=(1-PLR)^K$ \\
        \end{tabular}
    \end{center}

    TCP/IP assumes Data Link layer provides error-free packets with possible packet loss (but very low FER).
    End-to-End is used in most cases, implemented in Transport or Application;
    in lossy channels (e.g. wireless) link-by-link is implemented in Data Link. 
\end{minipage}

\vspace{-0em}\rule{\textwidth}{1.0pt}\vspace{-0em}

\begin{minipage}{0.1\textwidth}
    \section*{Delay models}
\end{minipage}%
\begin{minipage}{0.9\textwidth}
    \begin{tabular}{@{}l | l | l@{}}
        \textbf{Multiplex strategies}        & \textbf{Description}                                                                      & $T_{frame}$ \\ \hline
        Statistical              & Transmitted on first-come first-served basis                                              & $L/C$       \\ \hline
        Freq. division (FDM) & Link capacity $C$ divided into $m$ channels, each with bandwidth $W/m$ and capacity $C/m$ & $Lm/C$      \\ \hline
        Time division (TDM)      & Link capacity $C$ divided into $m$ channels in the time axis, each with capacity $C/m$    & $Lm/C$      \\
    \end{tabular}
\end{minipage}

\begin{minipage}{5.5em} \subsection*{Little's theorem} \end{minipage}
\begin{minipage}{7.0em}
    $N = \lambda \cdot T$ \\
    $N_W = \lambda \cdot T_W$
\end{minipage}
\begin{minipage}{48.5em}
    The time a client waits on queue $T_W$ depends only on the \# of clients in queue $N_W$ and client arrival rate $\lambda$, but not on the service rate (!).
\end{minipage}

\begin{minipage}{0.289\textwidth}
    \subsection*{Delay modelled as queue networks}
    Poisson arrivals can be described by a Poisson distribution with $\expected[A]=1/\lambda$, $\variance[A]=1/\lambda^2$.

    \textbf{Kendall notation:} A/S/s/K (A -- arrival stat. process; S -- service stat. proc.; s -- number of servers; K -- system buffer capacity

    \paragraph{Queues}
    $P(n)$ -- Prob. of Markov chain being in state $n$.
    $N = N_W + N_S$.
    $T = T_W + T_S$.

    Element being transmitted is the first in queue; queue is popped when that element is fully transmitted, meaning max queue time in M/M/1/B is $T_S (B-1)$.

\end{minipage}
\begin{minipage}{0.707\textwidth}
    \begin{tabular}{@{}l || c | c | c | c | c @{}}
        \textbf{M/M/1}   & $\displaystyle P(n) = \rho^n(1-\rho)$   & $\displaystyle N=\frac{\rho}{1-\rho}$ & $\displaystyle T= \frac{1}{\mu-\lambda}$ & $\displaystyle T_W = \frac{\rho}{\mu(1-\rho)}$ & $\displaystyle N_W = N-\rho$ \\ \hline
        \textbf{D/D/1}   &                                         & $\displaystyle N=\rho$                & $\displaystyle T=1/\mu$                  & $\displaystyle T_W = 0$                        & $\displaystyle N_W = 0$      \\ \hline
        \textbf{M/M/1/B} & \multicolumn{5}{c}{
            \begin{minipage}{8em}
                \vspace{-0.8em}
                \begin{alignat*}{2}
                    P(0) = \frac{1-\rho}{1-\rho^{B+1}} \\
                    P(n) = \rho^n\cdot P(0)
                \end{alignat*}
            \end{minipage}
            \begin{minipage}{12em}
                \vspace{-0.8em}
                \begin{equation*}
                    \rho = 1 \implies P(B) = \frac{1}{B+1}
                \end{equation*}
            \end{minipage}
            \begin{minipage}{16em}
                \vspace{-0.8em}
                \begin{equation*}
                    \rho \gg 1 \implies P(B) = \frac{\rho-1}{\rho} = \frac{\lambda-\mu}{\lambda}
                \end{equation*}
            \end{minipage}
        } \\ \hline
        \textbf{M/G/1}   &                                         & $\displaystyle N=N_W + \rho$                & $\displaystyle T=T_W+1/\mu$                             & $\displaystyle T_W = \frac{\lambda \expected[X^2]}{2(1-\rho)}$                    & $\displaystyle N_W = \lambda T_W$                 \\ \hline
        \textbf{M/D/1}   & \begin{minipage}{6.1em}$\expected[X]=1/\mu$\\$\expected[X^2]=1/\mu^2$\end{minipage} & $\displaystyle N=N_W + \rho$                & $\displaystyle T=T_W+1/\mu$                             & $\displaystyle T_W = \frac{\rho}{2 \mu (1-\rho)}$                    & $\displaystyle N_W = \frac{\rho^2}{2 (1-\rho)}$                 \\
    \end{tabular}
    \begin{minipage}[t]{0.557\linewidth}
        \textbf{M (Markovian)} -- Poisson process/exponential service time; \\
        \textbf{D (Degenerate)} -- Fixed inter-arrival interval/service time;
    \end{minipage}
    \begin{minipage}[t]{0.433\linewidth}
        \textbf{G (General)} -- General process, arrival/service times independent \& identically distributed, with given parameters $\expected[X]=1/\lambda$, $\expected[X^2]$
    \end{minipage}%
\end{minipage}

\begin{multicols}{2}
    \subsubsection*{Kleinroch independence approximation}
    Each link is a queue. Several streams merging on a link restore independence of interarrival times and packet lengths. Thus each link is M/M/1 if entry points have Poisson arrivals, packet lengths are nearly exponentially distributed, network is dense and has moderate/heavy traffic.

    Given each path $p$ on the network has a known arrival rate $x_p$, the arrival rate for link $i \leftrightarrow j$ is $\lambda_{ij} = \sum_{p : (i,j) \in p}{x_p}$; $\mu_{ij}$ is a known property of the link; all other things are obtained with M/M/1 equations. The total number $N$ of packets in the system is the sum of all $N_{ij}$, $\lambda = \sum{x_p}$ and $T$ is obtained from Little's theorem.

    \subsubsection*{Jackson networks}
    Now queues are not links, but nodes. Arrivals and services are Poisson, system is stable and has no cloggings ($\rho_j < 1$). For each node $j$ in a network with $K$ nodes, there are $r_j$ packets per second entering the system through $j$, and packets from other nodes; on leaving $j$, packets have prob $P_{ji}$ of going to $i$, and $1-\sum_{i=1}^{K}{P_{ji}}$ to leave system. Implies $\lambda_j = r_j + \sum_{i=1}^{K}{\lambda_i P_{ij}}$.

    \paragraph{Jackson's theorem} Let $\vec{n} = (n_1, n_2, ..., n_k)$ be a possible state; then $P(\vec{n}) = \prod_{j=1}^{K}{P_j(n_j)}$. Thus, all queues are independent; grab the Markov chain probs for M/M/1 queues, and we get $P(\vec{n}) = \prod_{j=1}^{K}{\rho_j^{n_j}(1-\rho_j)}$.

\end{multicols}

\vspace{-1.5em}\rule{\textwidth}{1.0pt}\vspace{-1.0em}

\begin{multicols}{3}
    \section*{MAC sublayer}
    \subsection*{Random Access Protocols}
    \paragraph{Aloha     }
    Sends packet to the networks, waits for a round-trip time, and if no ACK arrived then delay the packet sending by a random time to 'try later'. Pure: $S=Ge^{-2G}$, $S_{max}=1/2e=\SI{18.4}{\percent}$; Slotted: $S=Ge^{-G}$, $S_{max}=1/e=\SI{36.8}{\percent}$.

    \paragraph{CSMA      }
    Carrier Sense Multiple Access, first listens if there is traffic, only transmits if the channel is sensed free.

    \paragraph{CSMA Pers.}
    If busy, waits until medium becomes free and then transmits (persistent).

    \paragraph{CSMA NonP.}
    If busy, waits a random time and repeats (non-persistent).

    \paragraph{CSMA p-Pe.}
    If free, transmits with prob $p$, or defers transmission to next slot with prob $1-p$.

    \paragraph{CSMA/CD   }
    Collision detection; similar to CSMA Persistent but if a collision is detected \textbf{during} transmission, it is aborted and retransmission is delayed using a Binary Exponential Backoff algorithm (if there were $i$ consecutive collisions, transmit in a random slot picked from slot set $[0, 2^i-1]$). Does not use ACKs. Slots have size $T_{slot}=2T_{prop}$. $\lim_{N \rightarrow\infty}{S}=1/(1+3.44a)$. $T_f > \text{RTT}$ ensures all stations see collision.

    \paragraph{CSMA/CA} 
    Waits some DIFS time to check if no-one is transmitting, and then transmits; if someone is transmitting, then the current station waits DIFS plus a random backoff and tries again.

    \subsection*{Taking turns}
    The switch polls stations and decides which station gets to transmit first; or stations pass around tokens sequentially.

    \subsection*{Switches}
    Data Link device, forwards Ethernet frames, is transparent to hosts (hosts are unaware of the switch presence, as if hosts were directly connected). It has a forwarding table and is self-learning: when it receives a frame, it matches the incoming frame MAC address to the switch interface the frame arrived to. This differs from ARP, as ARP matches IP to MAC.
\end{multicols}

\vspace{-1.5em}\rule{\textwidth}{1.0pt}\vspace{-1.0em}

\begin{multicols}{3}
    \section*{Network layer}
    Provides two services: datagram network (connectionless service), and virtual circuit (connection-oriented, router maintains states, has forwarding tables).

    \begin{tabular}{@{}l | l@{}}
        \textbf{Description} & \textbf{Special IP address} \\ \hline
        This host                 & \texttt{0.0.0.0} \\
        Broadcast local network   & \texttt{255.255.255.255} \\
        Broadcast distant network & \texttt{<network>1111...} \\
        Loopback                  & \texttt{127.<anything>}
    \end{tabular}

    \paragraph{Internet Protocol (IP)}
    Can segment large packets, tho TCP already segments packets so IP only segments oversized UDP/ICMP datagrams.

    \paragraph{Address Resolution Protocol (ARP)}
    IP to MAC (can also get IP from MAC). When not on ARP table, device broadcasts ARP request; only device with that IP answers with corresponding MAC.

    \paragraph{Dynamic Host Configuration Protocol (DHCP)}
    Dynamically get IP address from a server, allowing IP address reuse. Client broadcasts \texttt{DISCOVER} to find all DHCP servers; all DHCP servers \texttt{OFFER} an IP each; client accepts one of the offers by \texttt{REQUEST}'ing the offered IP from one of the DHCP servers; server \texttt{ACK}s.

    \paragraph{Network Address Translation (NAT)}
    Separates private network from public network. Requests are rewritten to bear the NAT public address if going outside, or the dereferenced private address if going inside.

    \paragraph{Internet Control Message Protocol (ICMP)}
    Error/control messages. Used e.g. by \texttt{ping}. ICMP packets are encapsulated in IPv4 packets.

    \paragraph{IPv6}
    $\SI{128}{\bit}$ addresses.
    \textbf{Link-local:} used for comms in same LAN/link, not forwarded by routers.
    \textbf{Global unicast:} global address.
    \textbf{Anycast:} packet is received by any (only 1) member of group.
    \textbf{Multicast:} packet received by all group members.
    \textbf{Neighbor Discovery protocol (ND):} replaces ARP IPv4, ICMP router discovery, ICMP redirect.
\end{multicols}

\vspace{-1.5em}\rule{\textwidth}{1.0pt}\vspace{-0.0em}

\begin{minipage}[c]{0.603\textwidth}

    \begin{minipage}{0.14\linewidth}
        \section*{Trans-port}
    \end{minipage}%
    \begin{minipage}{0.86\linewidth}
        \begin{tabular}{@{}l | p{23.2em} | p{5.7em}@{}}
            \textbf{Pr.} & \textbf{Description} & \textbf{Used by} \\ \hline \hline
            UDP & Unreliable, connectionless, direct interface to IP with minimal protocol overhead & DNS, SNMP, DHCP\\ \hline
            TCP & Reliable (uses ARQ mechanism), connection-oriented, full-duplex, avoids receiver/network congestion & FTP, HTML
        \end{tabular}
    \end{minipage}
    \vspace{-1.5em}
    \begin{multicols}{2}
        \textbf{TCP} uses a system similar to \textbf{Go Back N}, except when both ends support selective acknowledgement (SACK), in which case a variation of Selective Repeat is used. Connection is established using \textbf{three-way handshake} (SYN, SYN/ACK, ACK). Data is interpreted as a \textbf{byte stream}, and packets are numbered by the sequence number of the first byte of data the packet carries. TCP connection is \textbf{full-duplex}, and each direction has different sequence numbers.

        \subsection*{Retransmissions}

        \paragraph{Adaptative ret.}
        $\text{RTT}=a \cdot \text{RTT} + (1-a) \text{RTT}_{\text{sample}}$, $\text{Timeout}=2\text{RTT}$

        \paragraph{Karn-Partridge alg.}
        RTT not updated on retransmissions, only updated with unambiguous ack's (ack's for segments that were only sent once). If there is a sharp increase in RTT, a new method is used: if a timeout occurs and causes a retransmission, timeout is doubled.
    \end{multicols}
\end{minipage}
\begin{minipage}[c]{0.39\textwidth}
    This avoids the possibility that TCP does not update RTT because there are only retransmissions, and as such would block into a very small RTT.

    \subsection*{Flow control}

    The receiver regularly broadcasts its receiver window \texttt{RWND}.
    Congestion window \texttt{CWND} is only kept/updated by the sender, and is used to implement congestion control (maximize efficiency, fairly distribute bandwidth).

    \paragraph{Additive Inc./Multip. Dec.}
    For each RTT, increment \texttt{CWND}; on timeout, divide \texttt{CWND} by 2. Sawtooth behavior.

    \paragraph{Slow Start}
    Start with small \texttt{CWND}, and double it for each RTT; when segment is lost (detected by timeout), a threshold is defined at half \texttt{CWND}, then \texttt{CWND} is reset, increases exponentially until it reaches threshold, and then increases linearly (congestion avoidance phase)
\end{minipage}

\vspace{-0em}\rule{\textwidth}{1.0pt}\vspace{-1.0em}

\begin{multicols}{3}

    \section*{Routing}
    Forwarding is to just use a forwarding table, and redirect a request to a certain node;
    routing is about computing the whole path of a request, and send it to the best node possible.

    \paragraph{Shortest-path routing}
    Destination-based, load-insensitive, uses minimum \# of hops or sum of link weights.

    \paragraph{Detecting topology changes} Usually a periodic \texttt{hello} message is used in both directions.

    \subsection*{Link-state routing}
    Station $u$ broadcasts a list of its links (the LSA -- link-state advertisement) to its neighbors. When a receiving station gets a more recent version of $u$'s LSA it re-broadcasts to its neighbors; thus the network is flooded with $u$'s LSA. When all stations know all links, each performs Dijkstra to create the routing table.

    \subsection*{Distance vector}
    Station $u$ keeps a table of least distance to all nodes $v$ and the link that should be used, $D(u,v), \forall v$. $u$ periodically informs its neighbors of its distance information $D(u,v), \forall v$; and each of its neighbors $v$ update their own distance vectors $D(v,w)$ by checking if passing through $u$ to reach $w$ is better (i.e., if $D(v,u)+D(u,w) < D(v,w)$). Initially all distances are assumed infinite (except each station's neighbors), and the distance vector will eventually converge. Distributed version of Bellman-Ford, it is iterative and asynchronous (triggered by receiving distance vector).

    \paragraph{Routing Information Protocol (RIP)}
    Distance vectors sent every $\SI{30}{\second}$ or when link cost changes. Does not scale well to large networks.

    \paragraph{Border Gateway Protocol (BGP)}
    Also known as Exterior Gateway Routing Protocol (eBGP), usually for high-level networks in hierarchy (e.g. inside an ISP, or between ISPs).

    \subsection*{Ethernet}
    Requires the network to be modelled as a tree, otherwise frames could loop forever. Uses a minimum spanning tree algorithm to build said tree, rooted in smallest ID switch.

    \subsection*{Flow network model}
    Networks are can be modelled as a flow network; packet networks are queue networks, as such this approach can only be used to estimate maximum data flow from one station to another.

\end{multicols}

\vspace{-1.5em}\rule{\textwidth}{1.0pt}\vspace{-0.5em}

\begin{center}
    \begin{minipage}{0.55\textwidth}
        \section*{Variables}
        \begin{tabular}{@{}l l | l@{}}
            $C$        & $[\SI{}{\bit/\second}  ]$ & Max. theoretical channel capacity                                        \\
            $B$        & $[\SI{}{\hertz}        ]$ & Channel bandwidth                                                        \\
            $f_s$      & $[\SI{}{symbol/\second}]$, $[\SI{}{baud}             ]$ & Sampler frequency; baudrate ($f_s=2 B$)    \\
            $P_r$      & $[\SI{}{\watt}         ]$ & Signal power seen by receiver                                            \\
            $N_0$      & $[\SI{}{\watt/\hertz}  ]$ & Spectral density of white noise power                                    \\
            $\lambda$  & $[\SI{}{pac /\second  }]$, $[\SI{}{\bit/\second     }]$ & Client arrival rate                        \\
            $\mu$      & $[\SI{}{pac /\second  }]$, $[\SI{}{\bit/\second     }]$ & Service rate                               \\
            $\rho$     & $[1                    ]$ & Traffic intensity; $\rho = \lambda/\mu$                                  \\
            $T_{prop}$ & $[\SI{}{\second}       ]$ & Propagation time from send. to rec.                                      \\
            $T_f$      & $[\SI{}{\second}       ]$ & Time to transfer a frame                                                 \\
        \end{tabular}
    \end{minipage}
    \begin{minipage}{0.44\textwidth}
        \begin{tabular}{@{}l l | l@{}}
            $P[A=k]$   & $[1                    ]$ & Prob. of frame requiring $k$ attempts                                    \\
            $E[A]$     & $[1                    ]$ & Expected \# of attempts                                                  \\
            $a$        & $[1                    ]$ & Ratio of $T_{prop}$ to $T_f$                                             \\
            $p_e$      & $[1                    ]$ & Frame error probability                                                  \\
            $S$        & $[1                    ]$ & Efficiency                                                               \\
            $W$        & $[1                    ]$ & Maximum window size                                                      \\
            $M$        & $[1                    ]$ & Modulo of sequence numbers                                               \\
            $N$        & $[1                    ]$ & Avg client \# in the system                                              \\
            $N_W$      & $[1                    ]$ & \# of clients in queue                                                   \\
            $N_S$      & $[1                    ]$ & \# of clients being served                                               \\
            $T$        & $[\SI{}{\second}       ]$ & Avg delay experienced by client                                          \\
        \end{tabular}
    \end{minipage}
\end{center}

\end{document}
