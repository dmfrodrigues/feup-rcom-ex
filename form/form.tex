\documentclass{form}
\usepackage{hyperref}
\author{Diogo Rodrigues (\texttt{\href{mailto:up201806429@fe.up.pt}{up201806429@fe.up.pt}})}
\title{RCOM -- Form}
\renewcommand\rightmark{
    Licensed under \href{https://creativecommons.org/licenses/by-nc-nd/4.0/}{CC BY-NC-ND 4.0 \vcenteredinclude{by-nc-nd-eu}}
}
% Document
\begin{document}
\noindent%\rule{\textwidth}{1.0pt}
\begin{minipage}{0.62\textwidth}
    \section*{Networks}
    \begin{tabular}{@{}p{16mm} | p{21mm} | l | p{19mm}@{}}
        \textbf{Inter-processor distance} & \textbf{Processors in same} & \textbf{Type of network} & \textbf{Examples} \\ \hline
        $\SI{    1}{     \meter}$ & Square meter & Personal Area Network (PAN)               & Bluetooth \\ \hline
        $\SI{   10}{     \meter}$ & Room         & \multirow{3}{*}{Local Area Network (LAN)} & \multirow{3}{19mm}{Over switch network} \\ \cline{1-2}
        $\SI{  100}{     \meter}$ & Building     &                                           & \\ \cline{1-2}
        $\SI{    1}{\kilo\meter}$ & Campus       &                                           & \\ \hline
        $\SI{   10}{\kilo\meter}$ & City         & Metropolitan Area Network (PAN)           & Cable TV \\ \hline
        $\SI{  100}{\kilo\meter}$ & Country      & \multirow{2}{*}{Wide Area Network (PAN)}  & \multirow{2}{*}{ISP network} \\ \cline{1-2}
        $\SI{ 1000}{\kilo\meter}$ & Continent    &                                           & \\ \hline
        $\SI{10000}{\kilo\meter}$ & Planet       & Internet                                  & \\
    \end{tabular}
\end{minipage}%
\begin{minipage}{0.38\textwidth}
    \section*{Physical layer}
    \textbf{Nyquist theorem}
    A sampler operating at frequency $f_s$ can completely reconstruct a signal with bandwidth $B$ when
    $f_s > 2B$
    
    \vspace{0.5em}

    \textbf{Nyquist bitrate}
    The theoretical maximum capacity of a noiseless channel with $M$ signal levels is $C = 2 B_c \log_2{M}$
    
    \begin{tabular}{@{}p{31mm} p{40mm}@{}}
        \textbf{Shannon's Law} & $\displaystyle C = B_c \log_2{\left(1+\frac{P_r}{N_0 B_c}\right)}$
    \end{tabular}
\end{minipage}

\vspace{0.5em}\noindent\rule{\textwidth}{1.0pt}

\noindent
\begin{minipage}{0.052\textwidth}
    \subsection*{OSI}
\end{minipage}%
\begin{minipage}{0.948\textwidth}
    \begin{tabular}{@{}p{9mm} | l | l | l | l | l@{}}
        \multicolumn{3}{c|}{\textbf{Layer}}                    & Data unit                & Function                                                        & Protocols          \\ \hline
        \multirow{4}{9mm}{Host layers}      & 7 & Application  & \multirow{3}{*}{Data}    & High-level APIs (resource sharing, remote file access)          &                    \\ \cline{2-3} \cline{5-6}
                                            & 6 & Presentation &                          & Data translation (encoding, compression, encrypt/decrypt)       &                    \\ \cline{2-3} \cline{5-6}
                                            & 5 & Session      &                          & Managing communication sessions                                 &                    \\ \cline{2-6}
                                            & 4 & Transport    & Segment                  & Reliable segment transmission (segmentation, ACK)               & TCP,UDP            \\ \hline
        \multirow{3}{9mm}{Media layers}     & 3 & Network      & Packet                   & Multi-node network (addressing, routing, traffic control)       & IP                 \\ \cline{2-6}
                                            & 2 & Data link    & Frame                    & Reliable frame transmission between two nodes                   & ARP,MAC,Ethernet   \\ \cline{2-6}
                                            & 1 & Physical     & Bit, symb.               & Communication of raw bits over physical medium                  &                    \\
    \end{tabular}
\end{minipage}%

\vspace{-0em}\noindent\rule{\textwidth}{1.0pt}\vspace{0.5em}

\noindent%\rule{\textwidth}{1.0pt}
\begin{minipage}[c]{0.495\textwidth}
    \section*{Data link layer}
    \subsection*{Automatic Repeat Request (ARQ)}
    % \subsubsection*{Stop\&Wait}
    \textbf{Stop\&Wait} |
    Each received frame must be ACK, sender only sends next frame if all previous frames were acknowledged.
    \begin{center}
        \begin{tabular}{c c}
            $\displaystyle a = \frac{T_{prop}}{T_f}$ & $\displaystyle P[A=k] = p_e^{k-1} (1-p_e)$ \\
            \multicolumn{2}{c}{$\displaystyle \expected[A] = \sum_{k=1}^{\infty}{k \times P[A=k]} = \frac{1}{1-p_e}$} \\
            \multicolumn{2}{c}{$\displaystyle S            = \frac{T_f}{E[A](T_f+2T_{prop})} = \frac{1}{E[A](1+2a)} = \frac{1-p_e}{1+2a}$}
        \end{tabular}
    \end{center}
    % \subsubsection*{Go Back N}
    \textbf{Go Back N} |
    Allows transmission before previous frames were ACK. It sends ACK(NR) meaning it acknowledges all packets with $index < NR$.
    If an out-of-sequence frame is received, REJ is sent with the expected frame number.

    \noindent
    \begin{minipage}[c]{28mm}
        $W = M-1$ \\ $= 2^k-1$
    \end{minipage}%
    \begin{minipage}[c]{71mm}
        \begin{equation*}
            S = \begin{dcases}
                \frac{1-p_e}{1+2ap_e}               & : W \geq 1+2a \\
                \frac{W(1-p_e)}{(1+2a)(1-p_e+Wp_e)} & : W < 1+2a
            \end{dcases}
        \end{equation*}
    \end{minipage}
\end{minipage}
\begin{minipage}[c]{0.495\textwidth}
    % \subsubsection*{Selective Repeat}
    \textbf{Selective Repeat} |
    Similar to Go Back N, but receiver accepts out-of-sequence frames and notifies the sender to send missing frames only.

    \begin{minipage}{60mm}
        \begin{equation*}
            S = \begin{dcases}
                1-p_e                 & : W \geq 1+2a \\
                \frac{W(1-p_e)}{1+2a} & : W < 1+2a
            \end{dcases}
        \end{equation*}
    \end{minipage}
    \begin{minipage}{30mm}
        \begin{equation*}
            W = M/2 = 2^{k-1}
        \end{equation*}
    \end{minipage}

    \subsection*{Reliability in TCP/IP reference model}
    $C$ - link capacity; $PLR$ - Packet loss ratio; $K$ - number of links between sender and receiver; assume all links have same $C$ and $PLR$.
    \begin{center}
        \begin{tabular}{@{}p{24mm} | p{60mm}}
            \textbf{Strategy} & \textbf{Description} \\ \hline
            Link-by-Link & \multirow{4}{60mm}{On error, the station closest to the sender notifies it. Repairs losses link by link. More complex, but better on very unreliable media.} \\ $S=1-PLR$ \\ \\ \\
            End-to-End   & \multirow{3}{60mm}{On error, the receiver notifies the sender. Less complex, but not acceptable on very unreliable media.} \\ $S=(1-PLR)^k$ \\ \\
        \end{tabular}
    \end{center}
\end{minipage}

\vspace{-0em}\noindent\rule{\textwidth}{1.0pt}\vspace{-2em}

\section*{Delay models}
% \subsection*{Multiplexing}
\begin{center}
    \begin{tabular}{@{}l | p{139mm} | l@{}}
        \textbf{Multiplexing strategies}        & \textbf{Description}                                                                      & $T_{frame}$ \\ \hline
        Statistical              & Transmitted on first-come first-served basis                                              & $L/C$       \\
        Frequency division (FDM) & Link capacity $C$ divided into $m$ channels, each with bandwidth $W/m$ and capacity $C/m$ & $Lm/C$      \\
        Time division (TDM)      & Link capacity $C$ divided into $m$ channels in the time axis, each with capacity $C/m$    & $Lm/C$      \\
    \end{tabular}
\end{center}

\noindent
\begin{minipage}[c]{0.640\textwidth}
    \subsection*{Delay modelled as queue networks}

    Poisson arrivals can be described by a Poisson distribution with $\expected[A]=1/\lambda$, $\variance[A]=1/\lambda^2$.

    \noindent
    \textbf{Kendall notation:} A/S/s/K (A -- arrival statistical process; S -- service statistical process; s -- number of servers; K -- capacity of the system in buffers)

    \noindent
    \begin{itemize}
        \setlength\itemsep{0em}
        \item \textbf{M (Markovian)} -- Poisson process/exponential service time;
        \item \textbf{D (Degenerate)} -- Fixed inter-arrival interval/service time;
        \item \textbf{G (General)} -- General process, arrival/service times are independent and identically distributed, with given parameters $\expected[X]=1/\lambda$, $\expected[X^2]$
    \end{itemize}
\end{minipage}
\begin{minipage}[c]{0.350\textwidth}
    \subsection*{Little's theorem}
    \vspace{-0.5em}
    \begin{tabular}{@{}c c@{}}
        $N = \lambda \cdot T$ &
        $N_W = \lambda \cdot T_W$
    \end{tabular}
    \vspace{0.5em}

    The time a client waits on queue $T_W$ depends only on the \# of clients in queue $N_W$ and the client arrival rate $\lambda$, but not on the service rate (!).
\end{minipage}%

\pagebreak

\noindent
\begin{minipage}{0.1\textwidth}
    \subsection*{Queues}
\end{minipage}%
\begin{minipage}{0.9\textwidth}
    $P(n)$ -- Probability of Markov chain being in state $n$.
\end{minipage}

\begin{center}
    \begin{tabular}{@{}l || c | c | c | c | c @{}}
        M/M/1   & $\displaystyle P(n) = \rho^n(1-\rho)$   & $\displaystyle N=\frac{\rho}{1-\rho}$ & $\displaystyle T=\frac{N}{\lambda} = \frac{1}{\mu-\lambda}$ & $\displaystyle T_W = T-T_S = \frac{\rho}{\mu(1-\rho)}$ & $\displaystyle N_W = T_W \lambda = N-\rho$ \\ \hline
        D/D/1   &                                         & $\displaystyle N=\rho$                & $\displaystyle T=1/\mu$                                     & $\displaystyle T_W = 0$                                & $\displaystyle N_W = 0$                    \\ \hline
        M/M/1/B & \multicolumn{5}{c}{
            \begin{minipage}{0.15\textwidth}
                \vspace{-0.8em}
                \begin{alignat*}{2}
                    P(0) = \frac{1-\rho}{1-\rho^{B+1}} \\
                    P(n) = \rho^n\cdot P(0)
                \end{alignat*}
            \end{minipage}
            \begin{minipage}{0.35\textwidth}
                \vspace{-0.8em}
                \begin{equation*}
                    \rho = 1 \implies P(B) = \frac{1}{B+1}
                \end{equation*}
            \end{minipage}
            \begin{minipage}{0.30\textwidth}
                \vspace{-0.8em}
                \begin{equation*}
                    \rho \gg 1 \implies P(B) = \frac{\rho-1}{\rho} = \frac{\lambda - \mu}{\lambda}
                \end{equation*}
            \end{minipage}
        } \\ \hline
        M/G/1   &                                         & $\displaystyle N=N_W + \rho$                & $\displaystyle T=T_W+1/\mu$                             & $\displaystyle T_W = \frac{\lambda \expected[X^2]}{2(1-\rho)}$                    & $\displaystyle N_W = \lambda T_W$                 \\ \hline
        M/D/1   & $\expected[X]=1/\mu$, $\expected[X^2]=1/\mu^2$ & $\displaystyle N=N_W + \rho$                & $\displaystyle T=T_W+1/\mu$                             & $\displaystyle T_W = \frac{\rho}{2 \mu (1-\rho)}$                    & $\displaystyle N_W = \frac{\rho^2}{2 (1-\rho)}$                 \\
    \end{tabular}
\end{center}

\vspace{-0.5em}\noindent\rule{\textwidth}{1.0pt}\vspace{-1.5em}

\section*{MAC sublayer -- Random Access Protocols}
\begin{center}
    \begin{tabular}{@{}p{21mm} | p{85mm} | p{30mm} | p{46mm}@{}}
        \textbf{Prot.} & \textbf{Description}                                                                                                                           & \textbf{Efficiency}                      & \textbf{Max. efficiency}                                                             \\ \hline
        ALOHA          & Sends packet to the networks, waits for a round-trip time, and if no ACK arrived then delay the packet sending by a random time to 'try later' & Pure: $S=Ge^{-2G}$; Slotted: $S=Ge^{-G}$ & Pure: $S_{max}=1/2e=\SI{18.4}{\percent}$; Slotted: $S_{max}=1/e=\SI{36.8}{\percent}$ \\
        CSMA           & \multicolumn{3}{p{170mm}}{Carrier Sense Multiple Access, first listens if there is traffic, only transmits if the channel is sensed free            } \\
        CSMA Pers.     & \multicolumn{3}{p{170mm}}{If busy, waits until medium becomes free and then transmits (persistent)                                                  } \\
        CSMA NonP.     & \multicolumn{3}{p{170mm}}{If busy, waits a random time and repeats (non-persistent)                                                                 } \\
        CSMA p-Pe.     & \multicolumn{3}{p{170mm}}{If free, transmits with probability $p$, or defers transmission to next slot with probability $1-p$                       } \\
        CSMA/CD        & \multicolumn{3}{p{170mm}}{Collision detection; similar to CSMA Persistent but if a collision is detected \textbf{during} transmission, it is aborted and retransmission is delayed using a Binary Exponential Backoff algorithm (if there were $i$ consecutive collisions, transmit in a random slot picked from slot set $[0, 2^i-1]$). Does not use ACKs. Slots have size $T_{slot}=2T_{prop}$. $\lim_{N \rightarrow\infty}{S}=1/(1+3.44a)$.} \\
        CSMA/CA        & \multicolumn{3}{p{170mm}}{Waits some DIFS time to check if no-one is transmitting, and then transmits; if someone is transmitting, then the current station waits DIFS plus a random backoff and tries again} \\
    \end{tabular}
\end{center}

\vspace{-0.5em}\noindent\rule{\textwidth}{1.0pt}\vspace{-0.0em}

\noindent
\begin{minipage}[c]{0.60\textwidth}
    % \section*{Network}

    \section*{Transport}
    \begin{tabular}{@{}l | p{82mm} | p{20mm}@{}}
        \textbf{Pr.} & \textbf{Description} & \textbf{Used by} \\ \hline \hline
        UDP & Unreliable, connectionless, direct interface to IP with minimal protocol overhead & DNS, SNMP, DHCP\\ \hline
        TCP & Reliable (uses ARQ mechanism), connection-oriented, full-duplex, avoids receiver/network congestion & FTP, HTML
    \end{tabular}

    \textbf{TCP} uses a system similar to \textbf{Go Back N}, except when both ends support selective acknowledgement (SACK), in which case a variation of Selective Repeat is used. Connection is established using \textbf{three-way handshake} (SYN, SYN/ACK, ACK). Data is interpreted as a \textbf{byte stream}, and packets are numbered by the sequence number of the first byte of data the packet carries. TCP connection is \textbf{full-duplex}, and each direction has different sequence numbers.

    \subsection*{Retransmissions}

    \textbf{Adaptative ret.} | 
    $\text{RTT}=a \cdot \text{RTT} + (1-a) \text{RTT}_{\text{sample}}$, $\text{Timeout}=2\text{RTT}$

    \textbf{Karn-Partridge alg.} | 
    RTT not updated on retransmissions, only updated with unambiguous ack's (ack's for segments that were only sent once). If there is a sharp increase in RTT, a new method is used: if a timeout occurs and causes a retransmission, timeout is doubled.
\end{minipage}
\begin{minipage}[c]{0.39\textwidth}
    This avoids the possibility that TCP does not update RTT because there are only retransmissions, and as such would block into a very small RTT.

    \subsection*{Flow control}

    The receiver regularly broadcasts its receiver window \texttt{RWND}.
    Congestion window \texttt{CWND} is only kept/updated by the sender, and is used to implement congestion control (maximize efficiency, fairly distribute bandwidth).

    \textbf{Additive Increase/Multiplicative Decrease} |
    For each RTT, increment \texttt{CWND}; on timeout, divide \texttt{CWND} by 2. \textbf{Sawtooth behavior}.

    \textbf{Slow Start} | 
    Start with small \texttt{CWND}, and double it for each RTT; when segment is lost (detected by timeout), a threshold is defined at half \texttt{CWND}, then \texttt{CWND} is reset, increases exponentially until it reaches threshold, and then increases linearly (\textbf{congestion avoidance} phase)
\end{minipage}

\noindent\rule{\textwidth}{1.0pt}\vspace{-0em}

\begin{center}
    \noindent%\rule{\textwidth}{1.0pt}
    \begin{minipage}{0.49\textwidth}
        \section*{Variables}
        \begin{tabular}{@{}l l | p{59mm}@{}}
            $C$           & $[\SI{}{\bit/\second}     ]$ & Max. theoretical capacity of a channel                      \\
            $B_c$         & $[\SI{}{\hertz}           ]$ & Channel bandwidth                                           \\
            $P_r$         & $[\SI{}{\watt}            ]$ & Signal power as seen by the receiver                        \\
            $N_0 B_c$     & $[\SI{}{\watt}            ]$ & Noise power as seen by the receiver                         \\
            $N_0$         & $[\SI{}{\watt/\hertz}     ]$ & Spectral density of white noise power                       \\
            $P_r/N_0 B_c$ & $[1                       ]$ & Signal to noise ratio (SNR)                                 \\
            $T_{prop}$    & $[\SI{}{\second}          ]$ & Propagation time from send. to rec.                         \\
            $T_f$         & $[\SI{}{\second}          ]$ & Time to transfer a frame                                    \\
            $a$           & $[1                       ]$ & Ratio of $T_{prop}$ to $T_f$                                \\
            $p_e$         & $[1                       ]$ & Frame error probability                                     \\
            $P[A=k]$      & $[1                       ]$ & Prob. of frame requiring $k$ attempts                       \\
            $E[A]$        & $[1                       ]$ & Expected number of attempts                                 \\
        \end{tabular}
    \end{minipage}
    \begin{minipage}{0.49\textwidth}
        \begin{tabular}{l l | p{59mm}}
            $S$           & $[1                       ]$ & Efficiency                                                  \\
            $W$           & $[1                       ]$ & Maximum window size                                         \\
            $M$           & $[1                       ]$ & Modulo of sequence numbers                                  \\
            $N$           & $[1                       ]$ & Average \# of clients in the system                         \\
            $N_W$         & $[1                       ]$ & \# of clients waiting on queue                              \\
            $N_S$         & $[1                       ]$ & \# of clients being served                                  \\
            $T$           & $[\SI{}{\second}          ]$ & Average delay experienced by a client                       \\
            $T_W$         & $[\SI{}{\second}          ]$ & Time a client waits on queue                                \\
            $T_S$         & $[\SI{}{\second}          ]$ & Service time                                                \\
            $\lambda$     & $[\SI{}{     \second^{-1}}]$ & Arrival rate of clients                                     \\
                          & $[\SI{}{\bit/\second^{-1}}]$ &                                                             \\
            $\mu$         & $[\SI{}{     \second^{-1}}]$ & Service rate                                                \\
                          & $[\SI{}{\bit/\second^{-1}}]$ &                                                             \\
            $\rho$        & $[1                       ]$ & Traffic intensity; $\rho = \lambda/\mu$                     \\
        \end{tabular}
    \end{minipage}
\end{center}

\end{document}
